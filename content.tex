\section{Introdução}

\begin{frame}[fragile]\frametitle{Introdução}

O conceito de Framework agrega além de códigos pré-prontos a utilização
de Padrões de Projeto. Um desses padrões, que nasceu junto com as
Frameworks é o Inversão de Controle, que mais tarde foi refinado por
Martin Fowler e ganhou o nome de Injeção de Dependência ou Dependency
Injection (DI).

\bigskip

\url{http://www.martinfowler.com/articles/injection.html}

\end{frame}

\section{Definição}

\begin{frame}\frametitle{Definição}

\textbf{Dependency Injection} significa dizer que os componentes de uma
classe não necessitam procurar por suas dependências num contexto global
ou usando localizadores de serviços. Ao invés disso suas dependências
são simplesmente injetadas na hora da criação e seu \textbf{Criador} é o
responsável por provê-las.

\end{frame}

\begin{frame}\frametitle{Definição}

Para isso são necessários três elementos:

\begin{itemize}
\item
  um \textbf{componente dependente},
\item
  uma \textbf{declaração das dependências} do componente e
\item
  um \textbf{componente injetor} que provê o componente dependente com
  suas dependências.
\end{itemize}
\end{frame}

\section{Vantagens}

\begin{frame}\frametitle{Vantagens}

Uma das razões que levaram a criação desse padrão é a necessidade de
testes automatizados.

Usando \textbf{DI} você é capaz de criar ambientes para teste e produção
devido ao \textbf{baixo acoplamento} proporcionado. Isso porque o padrão
conta com um \textbf{Criador} o qual configura o ambiente adequado
injetando as depedências necessárias.

\end{frame}

\section{Quem utiliza?}

\begin{frame}\frametitle{Quem utiliza?}

\begin{figure}
    \includegraphics[scale=0.3]{img/zend2.png}
    \caption{Zend Framework 2 (PHP)}
\end{figure}

\end{frame}

\begin{frame}\frametitle{Quem utiliza?}

\begin{figure}
    \includegraphics[scale=0.4]{img/oss-logo-spring.png}
    \caption{Spring (Java)}
\end{figure}

\end{frame}

\begin{frame}\frametitle{Quem utiliza?}

\begin{figure}
    \includegraphics[scale=0.4]{img/angularjs-logo.png}
    \caption{AngularJS (Javascript)}
\end{figure}

\end{frame}

\section{Exemplo}

\begin{frame}\frametitle{Exemplo}

Usando AngularJS vamos criar uma espécie de \emph{playlist} com músicas
de vários albúns no qual nós somos capazes de buscar por qualquer
música.

\end{frame}

\begin{frame}\frametitle{Como isso funciona?}

\inputjscodefile{src/injector-01.js}

\end{frame}

\begin{frame}\frametitle{Como isso funciona?}

\inputjscodefile{src/injector-02.js}

\end{frame}

\begin{frame}\frametitle{Como isso funciona?}

\inputjscodefile{src/injector-03.js}

\end{frame}

\begin{frame}\frametitle{Como isso funciona?}

\inputjscodefile{src/controller-01.js}

\end{frame}

\begin{frame}\frametitle{Como isso funciona?}

\inputjscodefile{src/controller-02.js}

\end{frame}

\begin{frame}[fragile]\frametitle{Como isso funciona?}

O método \texttt{map} da classe \texttt{Array} funciona assim:

\inputjscodefile{src/ex-array-map.js}

\end{frame}

\begin{frame}[fragile]\frametitle{Links}

\begin{itemize}
\item
  \url{http://framework.zend.com/manual/2.0/en/modules/zend.di.introduction.html}
\item
  \url{http://javafree.uol.com.br/artigo/871453/}
\item
  \url{http://pt.wikipedia.org/wiki/Inje%C3%A7%C3%A3o_de_depend%C3%AAncia}
\item
  \url{https://docs.angularjs.org/guide/di}
\item
  \url{http://martinfowler.com/articles/injection.html}
\item
  \url{http://googletesting.blogspot.com.br/2008/10/dependency-injection-myth-reference.html}
\item
  \url{http://misko.hevery.com/2008/09/10/where-have-all-the-new-operators-gone/}
\item
  \url{http://ionicframework.com/blog/angular-mobile-dev/}
\item
  \url{https://github.com/btford/briantford.com/blob/master/jade/blog/understanding-di.md}
\end{itemize}
\end{frame}

\begin{frame}\frametitle{Frases}

\begin{quote}
Factory creates the server, but does not run it.

\end{quote}
\end{frame}
